\documentclass[man,donotrepeattitle,letter]{apa6}

\usepackage[american]{babel}
\usepackage{csquotes}
\usepackage[style=apa,sortcites=true,sorting=nyt,backend=biber]{biblatex}
\usepackage{float}
\floatstyle{boxed}
\restylefloat{figure}
\usepackage{graphicx}
\usepackage{lipsum}
\DeclareLanguageMapping{american}{american-apa}
\addbibresource{main.bib}


\title{GEOG281 Midterm - Reflections on Peter Preston \& Beginnings of Pacific Rim Formation}
\shorttitle{GEOG281 Midterm}

\author{Evan Louie}

\affiliation{University of British Columbia}

\abstract{The formation of the Pacific Rim has been the by product of many events occuring between the ``West'' and Pacific-Asia stemming from the early-mid 19\textsuperscript{th} century, primarily based in colonisations, wars, trade disputes. These events have lead to many sterotypes and beliefs held by the West about Pacific-Asia; Peter Preston gives an illustrative view of the historical socio-political culture which Pacific-Asia has held until the present which gives a much more grounded explanation for Pacific-Asia's political culture than those offered by early 19\textsuperscript{th} century Western perception.}

\keywords{GEOG281, midterm, Pacific Rim, Pacific-Asia, West, Japan, China, Korea, United States, British Empire, relations, sterotypes, formation, integration}

\authornote{This document answers questions (1) \& (4) for the GEOG281 Midterm Examination}
\begin{document}
\maketitle

\tableofcontents
\newpage
\section{Reflections on Peter Preston}

\subsection{Illustration of Pacific-Asia}
The perception of Pacific-Asians and their values by the West have changed vastly over the past two centuries.  From the notion of the \textit{Yellow Peril}, coined by Kaiser Wilhelm II of Germany in the 1895, describing the emperor's fear of enslavement from the East, to the subsequent anti-Japanese/pro-Chinese propaganda of the early-mid 20th century; The West's outlook on Pacific-Asia and its inhabitants have changed tremendously do to the many events which have occurred between it and its neighbours over the pacific.  This is not to say that the fundamental cultural underpinnings of Pacific-Asia have changed as drastically as Western perception, but simply the way in which the West views them. In Peter Preston's \textit{Pacific Asia in the Global System}, Preston offers ``a speculative illustration of the political culture of Pacific-Asianess'' (Preston, 129), going on to describe Pacific-Asia's political culture in six parts:

\begin{enumerate}
  \item Economy is state-directed and policy is oriented to the pragmatic pursuit of growth
  \item State-direction is top-down style and pervasive in its reach
  \item Society is familial and thereafter oriented to the community
  \item Order is secured by pervasive social control machineries and a related hegemonic common culture
  \item Political debate and power is typically reserved to an elite sphere
  \item Culture comprises a mix of officially sanctioned tradition, which stresses consensus and eschews open conflict, and market sanctioned consumption
\end{enumerate}

\subsection{Historically Drive \& Resonance with Today}
Historically driven and even resonating with the ``Asian Values'' notion promulgated by Lee Kuan Yew, the former prime minister of Singapore in the 1990s, it becomes apparent that Preston's deductions are not without reason; And do in fact share many commonalities with the racial stereotypes held by the West in the late 19\textsuperscript{th} and early 20\textsuperscript{th} centuries.  However, looking towards the modern age, it becomes apparent that many of these cultural conclusions are quite debatable. As with South Korea and Taiwan's transition to democracy (Hu, 2005, pp.26-27), these two countries serve as examples, which to some degree, go against Preston's findings.  This is not to say that all of Preston's findings are false, Pacific-Asian people are still considered incredibly family-centric when compared to those of the West; as shown with South Korea's, often family run, zaibatsu like conglomerates, chaebol (The Economist, 1998).

\subsection{Comparing to Nineteen and Twentieth Centuries}
In the late 19\textsuperscript{th}  century, the notion of the \textit{Yellow Peril} become popular in the West with the mass immigration of Chinese coolie workers to various Western countries, most noticeably the United States.  Eventually becoming synonymous with the Japanese during the early-mid 20\textsuperscript{th}  century during Japan's military expansions, the term would eventually extend to all Asians of East-Asian decent.  Not heavily correlating to Preston's positings, the Yellow Peril and the West's racial views of Pacific-Asians were not founded in any sort of academic discourse, but out of fear. Characterizing Chinese men as violent job-stealing rapists, the Western perception of the Chinese in the late 19\textsuperscript{th}  century was mainly fuelled by the fear of job loss to the immigrating Chinese populous. As the Chinese were considered reliable workers willing to work in any condition without complaint (Norton, 1928)

The \textit{Racial Equality Proposal} was a proposal put forward by the Emperor of Japan at the Paris Peace Conference in 1919.  The proposal would, in short, ensure the equality of all nations regardless of race or culture (and thus, self governance); ironically, Japan originally sought equality for only itself with the West, but used language universalizing the proposal (Shimazu, 1998, p.115).  Suffering from unequal treaties after being forced to open to the West for trade, Japan was eager to establish themselves as political equals and correct such inequalities. It is at the Paris Peace Accord where the view of Pacific-Asians by the West best comes into focus.  Where, after receiving a majority vote of approval from member countries of the League of Nations, Chairman Wilson would veto the proposal in accordance to the British Empires disapproval (Temperley, 1924, p.352) and to satisfy his pro-segregation Southern Democrat constituents to ratify the final Treaty of Versailles.  With these actions in mind, the Western view of Pacific-Asia comes more into focus, with countries such as France, Italy, and Greece all approving of the proposal, the ``West'' had begun to see the need for universal equality among races, but was superseded by the wanting of the US and Britain.

\subsection{Asian Values}
In the 1990s, the political ideology of \textit{Asian Values} as made popular by Mahathir Mohamad and later Lee Kwan Yew, prime minister of Malaysia and leader of Singapore respectively. It outlined various commonalities shared amongst the Pacific-Asian countries to try and create a shared collective identity and fight against the push of democracy and free-market capitalism from the West (Barr, 2003).  The ideology would begin to wane in popularity after the 1997 Asian financial crisis, in which Asia lacked any mechanism to alleviate the crisis (Langguth, 2003).  The primary tenants of the Asian Values ideology were outlined in the Bangkok Declaration in 1993:

\begin{enumerate}
  \item Predisposition towards single-party authoritarian government
  \item Preference for social harmony
  \item Concern with socio-economic prosperity and the collective well-being of the community
  \item Loyalty and respect towards figures of authority
  \item Preference for collectivism and communitarianism
\end{enumerate}

It becomes apparent that the Asian Values ideology is extremely akin to Preston's illustration of Pacific-Asian political culture; so much so, that one could almost create a direct mapping between the tenants in both.  However, the Asian Values ideology is much more a direct affirmation of an Asiatic predisposition towards Communism and oligarchical/dictatorial leadership (De Bary, 1998), whereas Preston's illustration represents a historical summation moreover than some sort of biological/cultural inclination.

\subsection{Conclusion}
Preston's finding can best be viewed as an apt illustration of the historic view of Pacific-Asian political culture.  Ringing much more true than the racially charged views of the Western populous during the late 19\textsuperscript{th} and early-mid 20\textsuperscript{th} century, Preston effectively encapsulates the core principles of Pacific-Asian political cultures up to the present.  Whether or not the image drawn by Preston will always ring true to Pacific-Asia is unknown, nor is it known how much of an impact in/direct contact with the West has caused, however that does not matter; what is important, is the that the history surrounding Pacific-Asia has caused the region to develop in such a way that Preston's findings are affirmed.


\section{Historical Formation of the Pacific Rim}

\subsection{Introduciton}
Relationships within the Pacific Rim have not always been as friendly as they are today. Historically stemming from the colonial actions of the British Empire and the United States, these two super powers of the 19\textsuperscript{th} century would force open the door to Asiatic relations and trade; Catalyzing a series of events which would not only lead to colonisations and wars, but would ultimately end in the Pacific Rim known today.

\subsection{Opening of China}
The 17\textsuperscript{th}  and 18\textsuperscript{th}  centuries' marked an incredible growth in wealth for China, which, utilizing the Silk Road and isolationist policies, managed to stay almost wholly self-sufficient and not require many imported goods from other countries.  This lead to a massive inflow of European silver up until the early 19\textsuperscript{th}  century until the British East Indian Trading Company, utilizing Chinese middlemen, began to export opium to China.  This would lead to a noticeable outflow of silver and increase in opium addicts, which would go on to alarm the Chinese government.  The growing opium trade would eventually lead to two wars, the First and Second Opium Wars; The \textit{First Opium War} often being cited as the beginning of modern China (Janin, 1999, p.207) as well as the beginning of what Chinese Nationalists would eventually call the \textit{Century of Humiliation} (Kaufman, 2010, p.1-33).  With Hong Kong becoming a colony of the British Empire and the opening of numerous treaty ports (with even more after the \textit{Second Opium War}), these events marked the beginning of the relationship between the China and the West as well as the initial formation of the Pacific Rim.

\subsection{Opening of Japan}
Japan, akin to 18\textsuperscript{th}  century China, remained extremely isolationist in their trade policies.  Having limited trade with Dutch and Chinese ships, which required a special permit (Hawks, 2005), Japan had no intention of opening up their borders for trade with the West.  This all changed in 1852 when Commodore Perry docked at Uraga Harbour and demanded Japan to open trade with the United States, this being encapsulated in a letter from President Fillmore.  Powerless against the supremacy of the US navy, Perry and his fleet of \textit{Black Ships} would become the symbol Western technology and colonialism in Japan (Hawks, 2005).  Returning two years later with a fleet double the size of the original, Perry would sign into law the Convention of Kanagawa and officially open Japan to trade with the West.

\subsection{Preludes to War}
By the mid 19\textsuperscript{th}  century, China and Japan, Pacific-Asia's most industrious countries, had been forcibly opened to trade with the West.  Relations would continue to develop between China and the U.S, with many people emigrating from China to the U.S during the California Gold Rush and the construction of the Transcontinental Railroad.  Well received at first (Norton, 1928, pp.283-296), the U.S view of Chinese immigrants would eventually turn sour as competition for gold increased and the post-civil war economic boom ended (Kearny, 1878); eventually leading the \textit{Chinese Exclusion Act} in 1882.  Japanese would indirectly migrate to the U.S as emigration began to the, then, Kingdom of Hawaii; becoming part of the U.S in 1898, this would lead to the Japanese indirectly immigrating to the U.S.  Not wanting to anger the Japanese, who now sought colonization of China (and would thus hurt U.S trade), Japan and the U.S entered an informal agreement to ban emigration from Japan to the U.S, effectively indirectly copying the Chinese Exclusion Act (Weinberg, C, p.36).

\subsection{World Wars and Beyond}
With tensions high after World War I and Japan's \textit{Twenty-One Demands} of China, which laid claim to Germany's former holdings over China and economic power over Manchuria (Drea, 1998, pp.106-116); Friction would continue to arise between Japan and the U.S, as Japan would continue with extensive military colonization throughout Pacific-Asia.  Following a failed Second Sino-Japan War, Japan would continue in militaristic expansion and follow in pace with Germany in wanting to establish a Third Reich-esque body in Pacific-Asia; in what would eventually cumulate in Japan's formal declaration in creating the \textit{Greater East Asia Co-Prosperity Sphere} (De Bary, 2008, p.662).

Following the atomic bombings of Hiroshima and Nagasaki, Japan would officially surrender to the Allies and officially put an end to World War II. With the allies, primarily the United States, now occupying Japan, relations would start anew between the two countries; As the U.S reformed and rebuilt Japan economically and politically.  This would set the foundation for the current day relationship between the West and Japan as well as the setting of one of major connections within the Pacific Rim.

During World War II, China would develop a heavy communist presence in the form of the Communist Party of China.  Their presence would lead to an all out civil war between them and the Republic of China (Fenby, 2004); which in the end, would lead the establishment of the People's Republic of China in mainland China and the leaving of the Chinese nationalists to Taiwan. With the U.S considering communism a grave threat, the U.S would continue to fund Taiwan until officially recognizing the People's Republic of China as China's official government in 1979.

The Cold War represented an era of turmoil for Pacific Rim relations.  With China being a communist power and ally to the Soviet Union, China would eventually back communist North Korea in the Korean war (Ross, 2001). The U.S, backing South Korea, would ultimately engage in armed conflict with China until an armistice was signed, ending the Korean War. The Korean War would set the foundation for the relationships between North Korea and China, as well as South Korean and the United States, which exist today.

\subsection{Conclusion}
The relationships existing within the Pacific Rim have not always been as friendly as they are today.  Forged through colonisations, wars, and the wanting to trade, the socio-political relations which interconnect not only the countries within Pacific-Asia, but with those across the Pacific would not have occurred at all if not for the forced opening of Asiatic trade and relations by the British Empire and the United States.  Said colonisations, wars, and trade were in fact the foundation for the relationships which would come to be recognized as the Pacific Rim.



\nocite{*}
\printbibliography

\newpage
[This page is intentionally left blank.]

\end{document}

%
% Please see the package documentation for more information
% on the APA6 document class:
%
% http://www.ctan.org/pkg/apa6
%
